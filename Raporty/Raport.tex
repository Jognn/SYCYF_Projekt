\documentclass[a4paper,titleauthor]{mwart} 

\usepackage{polski}
\usepackage[utf8]{inputenc}
\usepackage{graphicx} %pakiet do wstawiania grafiki
\usepackage[hyphens]{url} %pakiet do wstawiania linkow
%\usepackage[hidelinks,breaklinks]{hyperref}
\usepackage{authblk}%pakiet do tworzenia afiliacji
\usepackage{tabularx}%pakiet do tabel
\usepackage[a4paper, left=2cm, right=2cm, top=3cm, bottom=3cm]{geometry}
\usepackage{listings}
\usepackage{placeins}%pakiet do kontroli umieszczania obiektow
\usepackage{hyperref}%pakiet do m.in. kolorowania linkow

\usepackage[tablegrid,owncaptions]{vhistory}
\renewcommand{\vhhistoryname}{Historia zmian}
\renewcommand{\vhversionname}{Wersja}
\renewcommand{\vhdatename}   {Data}
\renewcommand{\vhauthorname} {Autor}
\renewcommand{\vhchangename} {Opis zmian}

\renewcommand\figurename{Rys.}%skrocony podpis
\renewcommand\lstlistingname{Wydruk}


%------------------------------------------------------------------------
% Dane do strony tytułowej

\title{{\Huge  Projekt SYCYF}\\ - \\{\Large Zespół nr 6}\\ }

\author{Sofiia Levchenko \and Joanna Stalenczyk \and Julia Boryslawska \and Jakub Tarczynski \and Jan Zobniow}

\date{\today}

%------------------------------------------------------------------------
% Początek dokumentu
\begin{document}

%Automatycznie generowany tytuł dokumentu
\maketitle
%------------------------------------------------------------------------
% Automatycznie generowany spis treści
\tableofcontents

%------------------------------------------------------------------------
\section{Wstęp}
\label{sec:wstep}%etykieta

Raport bedzie dokumentowany \textbf{przyrostowo} zgodnie z realizacja projektu. Poszczególne etapy realizacji projektu obejmują: 

\renewcommand{\labelenumi}{\Roman{enumi}}
\begin{enumerate}\setlength{\itemsep}{0.2\baselineskip} 
	\item Etap wstępny – stworzenie zespołu i organizacja warsztatu pracy, 
	\item Etap zdobywania informacji – analiza literatury, istniejących metod, zebranie wiedzy teoretycznej związanej z tematem projektu, 
	\item Etap opracowania koncepcji – szukanie rozwiązań, najlepiej sprawdzi się proces burzy mózgów (mapy myśli), opracowanie koncepcji rozwiązania  na podstawie zdobytej wiedzy, opracowanie prostego modelu referencyjnego (Python, MATLAB/GNU Octave, itp) i danych do testowania  
	\item Etap implementacji – na tym etapie rozwijamy i rozbudowujemy koncepcje projektowe docelowego systemu, modelujemy elementy systemu w HDL, weryfikujemy funkcjonalnie, integrujemy i oceniamy prototypy, 
	\item Etap uruchomienia – wdrożenie projektu, uruchomienie na docelowej platformie, przetestowanie według wcześniej opracowanych scenariuszy testowych. 
\end{enumerate}

Prace wykonane w ramach każdego etapu beda opisane w oddzielnych rozdziałach raportu.

\section{Organizacja prac}
\label{sec:organizacja}

Rozdział ten będzie opisywał zadania zrealizowane w ramach Etapu\texttt{I}. W tym rozdziale będą omówione:

\begin{itemize}
	\item analiza podejścia Design Thinking oraz trzech jego wersji (pięć kroków Design Thinkig'u, trzy zachodzące na siebie fazy procesu Design Thinking'u oraz cztero-fazowy proces "Double Dimond")
	\item wybor jedynego sposobu zarządzania projektem
	\item analiza różnego rodzaju narzędzi (TeXstudio, Overleaf, Microsoft Word, Microsoft Teams, Skype, GitHub oraz GitLab)
	\item organizacja warsztatu pracy, dobór narzędzi (Overleaf, Microsoft Teams, GutHub, itp.)
	\item ostateczny dobór narzędzi
	\item organizacja warsztatu pracy
\end{itemize}

\subsection{Design Thinking}
\label{sec:design_thinking}
Co to jest "Design Thinking?" \newline
\newline
Design thinking – jest to proces odnoszący się do procesów poznawczych, strategicznych i praktycznych, dzięki którym koncepcje projektowe (propozycje nowych produktów, usług itp.) są opracowywane przez projektantów i / lub zespoły projektowe~\cite{DesignThinking1}. \newline \newline Wiele osób próbowało sformalizować podstawowe zasady zastosowania myślenia projektowego w organizacjach, za pomocą modeli procesów i metodologii. Pomimo niekończących się modyfikacji i mutacji istnieją trzy, trwałe, ogólne przyjęte modele~\cite{DesignThinking2}:

 \begin{itemize}
 	\item Pięć kroków Design Thinkig'u
 	\item Trzy zachodzące na siebie fazy procesu Design Thinking'u
 	\item Cztero-fazowy proces "Double Dimond"
 \end{itemize}

Diagramy i słownictwo mogą się różnić, ale we wszystkich modelach myślenia projektowego występują wyraźne tematy:

 \begin{itemize}
	\item \textbf{Zorientowane na człowieka}\newline \newline Fazy odkrywania i inspiracji koncentrują się na badaniach proponowanego użytkownika - "Kim oni są?", "Czego chcą / potrzebują?", "Jak się zachowują?" Chodzi o budowanie empatii przez zespół projektowy z użytkownikiem końcowym i zrozumienie, dla kogo projektują.\newline
	\item \textbf{Iteracyjny} \newline \newline Fazy opracowywania, dostarczania i wdrażania skupiają się na usuwaniu najsłabszych pomysłów i ulepszaniu najsilniejszych poprzez prototypowanie, testowanie i optymalizację.\newline
	\item \textbf{Interdyscyplinarne} \newline \newline Wszystkie modele pokazują podróż, która rozbiega się i zbiega, przynajmniej w początkowe rozwiązanie. Każda faza nie jest własnością jednego zespołu z ustalonymi zadaniami; zamiast tego wspólny zespół projektowy wyrusza razem w podróż. Należy zauważyć, że zespoły interdyscyplinarne różnią się od zespołów multidyscyplinarnych tym, że każdy z nich ma wspólną odpowiedzialność za projekt i jego sukces, zamiast opowiadać się za własną specjalizacją.
 \end{itemize}

Ponizej beda sa przedstawione skrócone opisy, etapy każdego z danych procesow, bądź ich wady oraz zalety, w wykorzystaniu dla zarzadzania projektem.

 \subsubsection{Pięć kroków Design Thinkig'u}


 \subsubsection{Trzy zachodzące na siebie fazy procesu Design Thinking'u}

Drugim w kolejce jest model, który nazywa się"Trzy zachodzace na siebie fazy procesu Design Thinking'u" ~\cite{Proces2}. Jest to proces, ktory skupiony jest na człowieku. \newline \newline 
Co to znaczy? \newline \newline
Rozpoczyna on się od osób, dla których projektujesz, a kończy na nowych rozwiązaniach, które są dostosowane i wygenerowane specjalnie do spełnienia ich potrzeb. Projektowanie zorientowane na człowieku polega na budowaniu głębokiej empatii z ludźmi, dla których projektujesz; generowanie tony pomysłów; budowanie wiązki prototypów; dzielenie się tym, co stworzyłeś, z ludźmi, dla których projektujesz; i ostatecznie wypuszczenie na rynek nowego innowacyjnego rozwiązania. \newline \newline
Konstrukcja skoncentrowana na człowieku, w danym procesie, składa się z trzech faz:\newline
 \begin{itemize}
 \item W fazie inspiracji dowiesz się o potrzebach bezpośrednio od ludzi, dla których coś projektujesz, zanurzając się w ich życiu. 
 \item W fazie idei zrozumiesz, czego się nauczyłeś, zidentyfikujesz możliwości projektowania i stworzysz możliwe rozwiązania. 
 \item Na etapie wdrażania wprowadzisz swoje rozwiązania w życie, a ostatecznie na rynek. Będziesz wiedział, że Twoje rozwiązanie odniesie sukces, ponieważ w centrum procesu są ludzie, którym chcesz służyć.
 
 Proces ten przedstawiony jest poniżej:

 \begin{figure}[h]
 	\centering
 	\includegraphics[width=0.8\textwidth]{2}
 	\caption{Trzy zachodzące na siebie fazy procesu Design Thinking'u}
 \end{figure}

\end{itemize}

Zalety danego modelu polegają na:\newline
\begin{itemize}
\item Ułatwia i przyspiesza przyjęcie rozwiązania: dzięki takiemu podejściu projektowanie oparte na Design Thinking zaczyna się od potrzeby użytkownika końcowego, kończy na zbudowaniu rozwiązania (technologicznego lub procesowego), które jest dla niego naprawdę cenne.
\item Silne zaangażowanie użytkownika końcowego, który czuje się zaangażowany już od strony projektowej: sposób dla firm na ulepszenie swoich pracowników i „wzięcie ich na pokład” stojącego przed nimi wyzwania;
\item Iteracyjne podejście na etapie projektowania pozwala przetestować propozycje rozwiązań i modyfikować je, aż do osiągnięcia najbardziej odpowiedniego rozwiązania przed przystąpieniem do faktycznego wdrożenia;
\end{itemize}
 Wady danego modelu w swoim czasie polegają na:\newline
\begin{itemize}
	\item Projekt moze trwać długo, a końcową werseję można zastosować tylko do ograniczonych celów
	\item Zastosowanie metodologii projektowania korporacyjnych rozwiązań cyfrowych może kolidować z ograniczeniami nałożonymi przez integrację z już używanymi rozwiazaniami.
	\item Myślenie projektowe wymaga bezpośredniego zaangażowania użytkowników, którzy muszą mieć możliwość wniesienia własnego wkładu (dostępność czasu i zasobów)
\end{itemize}

 \subsubsection{Cztero-fazowy proces "Double Dimond"}

\subsection{Zarządzanie projektem}
\label{sec:zarządzanie_projektem}

\subsubsection{Metody}
\label{sec:narzędzia}

\subsubsection{Narzędzia}
\label{sec:narzędzia}
\textbf{LaTeX}
\indent

Zalety:
\begin{itemize}

\item[-]
Składnia LaTeX-a jest używana jako wzorowa m. in. do narzędzi Libre, Open Office oraz na różnego rodzaju forach matematycznych i fizycznych;

\item[-]
Wszystkie tytuły sekcji / podpisy tabel, grafik, wykresów są jednakowo formatowane;

\item[-]
LaTeX automatycznie numeruje sekcje, rysunki, tabele, cytowania, przypisy itp.;

\item[-]
Program generuje spis treści;

\item[-]
Istnieją gotowe szablony dokumentów;

\item[-]
Pliki PDF generowane przy pomocy LaTeX-a są estetyczne wizualnie;

\item[-]
Możliwość tworzenia wykresów/diagramów za pomocą kodu, bez używania dodatkowych programów; 

\item[-]
LaTeX jest programem o wolnym dostępie – nie musimy za niego płacić;

\item[-]
Pliki LaTeX-a możemy otwierać przy pomocy wielu narzędzi. Możemy używać programów zainstalowanych na komputerze lub wykorzystywać programy internetowe. W przypadku tych drugich istnieje możliwość dzielenia projektu między wiele osób. 
\end{itemize}

\indent

Wady:
\begin{itemize}


\item[-]
Konieczność stosowania odpowiednich bibliotek;

\item[-]
Trudniejsza zmiana np. rodzaju czcionki, marginesów i innych parametrów formatowania pliku w porównaniu do narzędzi typu Word;

\item[-]
Komunikaty o błędach są często mało zrozumiałe;

\item[-]
Aby korzystać z LaTeX-a trzeba nauczyć się odpowiednich komend oraz poznać składnię.
\end{itemize}
\vspace{2cm}
\textbf{Word:}

\indent

Zalety:
\begin{itemize}

\item[-]
Program nie wymaga poznania specjalistycznej składni ani komend;
\item[-]
W łatwy sposób można zmienić parametry np. czcionki, marginesów;
\item[-]
Nie wymaga importowania odpowiednich bibliotek.
\end{itemize}

\indent

Wady:
\begin{itemize}

\item[-]
Wprowadzenie zapisów (symboli) matematycznych jest bardzo trudne;
\item[-]
 Występują problemy z formatowaniem (np. odpowiednie marginesy, akapity);
\item[-]
Podczas przesyłania pliku zdarza się jego deformacja.
\end{itemize}

------------------------------------------------------------
\section{Informacje podstawowe}
\label{sec:informacje_podstawowe}

\section{Koncepcja}
\label{sec:koncepcja}

\section{Implementacja}
\label{sec:implementacja}


\section{Uruchomienie}
\label{sec:uruchomienie}


\section{Podsumowanie}
\label{sec:podsumowanie}



\bibliographystyle{plabbrv} % plplain plabbrv plalpha
\begin{thebibliography}{SYCYfProjekt}
\bibitem{DesignThinking1}
 Visser, W. 2006, The cognitive artifacts of designing, Lawrence Erlbaum Associates.
\bibitem{DesignThinking2}
"Design Thinking: A Beginner’s Guide to the History, Terminologies and Methodologies" by Rhoda Sell \url{https://blog.prototypr.io/design-thinking-a-beginners-guide-to-the-history-terminologies-and-methodologies-e527f7afdcd1}
\bibitem{Proces1}
"The five steps of design thinking" by D.School, Stanford University’s Institute of Design \url{https://dschool-old.stanford.edu/sandbox/groups/designresources/wiki/36873/attachments/74b3d/ModeGuideBOOTCAMP2010L.pdf}
\bibitem{Proces2}
"The three overlapping phases of design thinking" by IDEO. \url{https://www.designkit.org/human-centered-design}
\bibitem{Proces3}
"The four-phased ‘Double Diamond’ process" by the Design Council \url{https://www.designcouncil.org.uk/news-opinion/what-framework-innovation-design-councils-evolved-double-diamond}
\end{thebibliography}

\end{document}

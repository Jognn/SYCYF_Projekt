%styl klasy z Polskimi Normami oprac. Marcin Wolinski
\documentclass[a4paper,titleauthor]{mwart} 

\usepackage{polski}
\usepackage[utf8]{inputenc}
\usepackage{graphicx} %pakiet do wstawiania grafiki
\usepackage[hyphens]{url} %pakiet do wstawiania linkow
%\usepackage[hidelinks,breaklinks]{hyperref}
\usepackage{authblk}%pakiet do tworzenia afiliacji
\usepackage{tabularx}%pakiet do tabel
\usepackage[a4paper, left=2cm, right=2cm, top=3cm, bottom=3cm]{geometry}
\usepackage{listings}
\usepackage{placeins}%pakiet do kontroli umieszczania obiektow
\usepackage{hyperref}%pakiet do m.in. kolorowania linkow

\usepackage[tablegrid,owncaptions]{vhistory}
\renewcommand{\vhhistoryname}{Historia zmian}
\renewcommand{\vhversionname}{Wersja}
\renewcommand{\vhdatename}   {Data}
\renewcommand{\vhauthorname} {Autor}
\renewcommand{\vhchangename} {Opis zmian}

\renewcommand\figurename{Rys.}%skrocony podpis
\renewcommand\lstlistingname{Wydruk}


%------------------------------------------------------------------------
% Dane do strony tytułowej

\title{{\Huge  Projekt SYCYF}\\ - \\{\Large Zespół nr 1}\\ }

\author{Jan Warszawski \and Piotr Krakowski \and Natalia Gdańska}

\date{\today}

%------------------------------------------------------------------------
% Początek dokumentu
\begin{document}
	
%Automatycznie generowany tytuł dokumentu
\maketitle
%podpis na dole strony, bez numeru


%------------------------------------------------------------------------


%------------------------------------------------------------------------
% Automatycznie generowany spis treści
\tableofcontents

%------------------------------------------------------------------------
\section{Wstęp}
\label{sec:wstep}%etykieta

Raport ma dokumentować \textbf{przyrostowo} realizację projektu. Poszczególne etapy realizacji projektu obejmują: 

\renewcommand{\labelenumi}{\Roman{enumi}}
\begin{enumerate}\setlength{\itemsep}{0.2\baselineskip} 
	\item Etap wstępny – stworzenie zespołu i organizacja warsztatu pracy, 
	\item Etap zdobywania informacji – analiza literatury, istniejących metod, zebranie wiedzy teoretycznej związanej z tematem projektu, 
	\item Etap opracowania koncepcji – szukanie rozwiązań, najlepiej sprawdzi się proces burzy mózgów (mapy myśli), opracowanie koncepcji rozwiązania  na podstawie zdobytej wiedzy, opracowanie prostego modelu referencyjnego (Python, MATLAB/GNU Octave, itp) i danych do testowania  
	\item Etap implementacji – na tym etapie rozwijamy i rozbudowujemy koncepcje projektowe docelowego systemu, modelujemy elementy systemu w HDL, weryfikujemy funkcjonalnie, integrujemy i oceniamy prototypy, 
	\item Etap uruchomienia – wdrożenie projektu, uruchomienie na docelowej platformie, przetestowanie według wcześniej opracowanych scenariuszy testowych. 
\end{enumerate}

Prace wykonane w ramach każdego etapu powinny być opisane w oddzielnych rozdziałach raportu. W tym dokumencie pokazano przykładową strukturę raportu. 

W opisach należy wykorzystywać rysunki, wykresy i tabele dla ułatwienia zrozumienia koncepcji Autorów. Na przykładzie rysunku  \ref{fig:double_diamond}  można zobaczyć, jak załączać rysunki i się do nich odwoływać w tekście. Podobnie tabela \ref{tab:tabela_prawdy} demonstruje użycie tabel, a wydruk \ref{lst:wydruk} jak wykorzystywać listingi kodu. 

Postępy prac należy odpowiednio opisać w historii zmian tak, aby opiekun projektu mógł śledzić wkład poszczególnych członków zespołu w realizację projektu.


\begin{table}[ht]
	\begin{center}
		\caption{Przykładowa tablica}
		\label{tab:tabela_prawdy}
		\begin{tabular}{|r||c|c||c|}
			\hline 
			PS & data=0 & data=1 & ready \\
			\hline \hline
			s0 & s1 & s0 & 0 \\
			s1 & s2 & s0 & 0 \\
			s2 & s2 & s3 & 0\\
			s3 & s1 & s4 & 0 \\ 
			s4 & s1 & s0 & 1 \\ 
			\hline
		\end{tabular}
	\end{center}
\end{table}

\begin{lstlisting}[label=lst:wydruk,caption={Przykładowy wydruk},language=Verilog,numbers=left]
module pattern_detec // 1100 
(	
input clk, reset,
input data,
output ready
);

reg [2:0] aut_reg,aut_next;

localparam s0=0, s1=1, s2=2, s3=3, s4=4;

always@(posedge clk, posedge reset)
if(reset)
aut_reg <= s0;
else
aut_reg <= aut_next;

always@*
case(aut_reg)
s0: if(data) aut_next = s0; else aut_next = s1; //0
s1: if(data) aut_next = s0; else aut_next = s2; //0
s2: if(data) aut_next = s3; else aut_next = s2; //1
s3: if(data) aut_next = s4; else aut_next = s1; //1
s4: if(data) aut_next = s0; else aut_next = s1;
default: aut_next = s0; 
endcase

assign ready = aut_reg == s4 ? 1'b1 : 1'b0;	

endmodule
\end{lstlisting}

	
%------------------------------------------------------------------------
\section{Organizacja prac}
\label{sec:organizacja}

Rozdział ten powinien opisywać zadania zrealizowane w ramach Etapu\texttt{I}. Można omówić:

\begin{itemize}
	\item podejście Design Thinking (w wersji Double Diamond lub innej),
	\item wybór sposobu zarządzania projektem
	\item organizacja warsztatu pracy, dobór narzędzi (Overleaf, Microsoft Teams, GutHub, itp.)
\end{itemize}

\subsection{Design Thinking}
\label{sec:design_thinking}

\subsection{Zarządzanie projektem}
\label{sec:zarządzanie_projektem}

\subsubsection{Metody}
\label{sec:narzędzia}

\subsubsection{Narzędzia}
\label{sec:narzędzia}

%------------------------------------------------------------------------
\section{Informacje podstawowe}
\label{sec:informacje_podstawowe}

Rozdział ten (i podrozdziały) powinien opisywać zadania zrealizowane w ramach Etapu\texttt{II}. W podrozdziałach należy przedstawić omówienie zagadnień niezbędnych (albo przydatne) do realizacji projektu, np.:

\begin{itemize}
	\item informacje znalezione w publikacjach, na stronach internetowych, itp.,
	\item istniejące rozwiązania,
	\item informacje teoretyczne niezbędne do realizacji projektu (ilustrowane równaniami, tabelami, przykładami, itp.),
	\item narzędzia, które zostaną użyte do realizacji projektu (Quartus, Vivado, Modelsim, Matlab, Octave, itp.).
\end{itemize}

Ten rozdział powinien być bogaty w odnośniki do bibliografii. Nie ma potrzeby wnikać w szczegóły, jeśli można się odwołać do materiałów, które je zawierają. Np. omawiając LaTeX po szczegóły można odesłać do \cite{WikiLatex}.

%------------------------------------------------------------------------
\section{Koncepcja}
\label{sec:koncepcja}

W tym rozdziale (i podrozdziałach) należy opisać zadania zrealizowane w ramach Etapu\texttt{III}. Może on zawierać mapę myśli ilustrującą burzę mózgów, która doprowadziła do opracowania koncepcji. Konieczne są natomiast:

\begin{itemize}
	\item schemat blokowy koncepcji rozwiązania i jego omówienie,
	\item przykład ilustrujący działanie,
	\item model referencyjny w Python, MATLAB/GNU Octave, itp),
	\item danych do testowania i opis scenariuszy testowych.
\end{itemize}



%------------------------------------------------------------------------
\section{Implementacja}
\label{sec:implementacja}

W tym rozdziale (i podrozdziałach) należy opisać sposób realizacji sytemu (Etap\texttt{IV}). Omówienie realizacji poszczególnych modułów (wydruki, wykresy symulacji, ocena szybkości działania i wymaganych zasobów). Rozdział powinien zawierać ocenę opracowanego rozwiązania (ocena szybkości działania i wymaganych zasobów). Należy zawrzeć opis weryfikacji funkcjonalnej przy użyciu testbench'a opracowanego z użyciem danych do testowania i scenariuszy testowych przedstawionych w rozdziale \ref*{sec:koncepcja}.

\section{Uruchomienie}
\label{sec:uruchomienie}

W tym rozdziale (i podrozdziałach) należy opisać uruchomienie systemu na docelowej platformie (Etap\texttt{V}). 
	
\section{Podsumowanie}
\label{sec:podsumowanie}

W tym rozdziale (i podrozdziałach) należy dokonać krytycznej oceny zaproponowanego rozwiązania i możliwości jego rozwoju. 


\bibliographystyle{plabbrv} % plplain plabbrv plalpha
\bibliography{SYCYF_proj}
	
\end{document}